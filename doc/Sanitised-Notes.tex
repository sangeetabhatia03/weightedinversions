\documentclass{article}
\usepackage[utf8]{inputenc}
\usepackage{amsmath,amssymb,amsthm}
\usepackage{algorithm2e}
\usepackage{natbib}
\usepackage{booktabs,tabularx}

\newtheorem{definition}{Definition}
\newtheorem*{lemma}{Lemma}
\newcommand{\newl}{\ell}
\begin{document}

\section{Problem}
\label{sec:problem}
We work on linear permutations on $n$ regions. Inversions are defined as acting on contiguous sets of preserved regions (defined by aligning several genomes of interest). If an inversion acts on $j$ regions we call it a $j$-inversion. A $j$-inversion is the permutation
\[t_{ij} := (i, j)(i+1, j-1) \hdots.\]
Let 
\[\Sigma := \{ t_{ij} \mid i < j \}.\]
Let $C$ be the set of Coxeter generators. Clearly $C \subset \Sigma$ i.e. $t_{i, i+1}$ is simply the Coxeter generator $s_i = (i, i + 1)$.  We use $\newl_C(\pi)$ to denote the Coxeter length of a permutation $\pi$.

Let $\omega : \Sigma \rightarrow \mathbb{R}^+$ be a function that assigns a positive weight to each inversion.
The problem we are working on is to find the weighted length of a permutation $\pi \in S_n$ over the generating set $\Sigma$ and weight function $\omega$. An associated problem is to find a minimal weight representative for $\pi$. The assumptions on the the weights are listed in section~\ref{sec:assm} and weighted length of a permutation is defined formally in definition~\ref{def:wwl}.

\section{Assumptions} \label{sec:assm}
\begin{itemize}
\item Define length of an inversion $t_{ij}$ to be $j - i + 1$. If $j -i = k - l$ then $\omega(t_{ij}) = \omega(t_{kl})$. All inversions of the same length carry the same weight.
\item $\omega(s_i) < \omega(t_{ij}) \quad \forall s_i \in C, t_{ij} \in \Sigma \setminus C$. That is, the weight assigned to the Coxeter generators is minimal.
\item $\omega(t_{ij}) \geq 1 \quad \forall t_{ij} \in \Sigma$.
%\item $\omega(t_{ij}) \leq \newl_C(t_{ij})$ where $\newl_C(t_{ij})$ is the Coxeter length of the inversion $t_{ij}$. The weight assigned to any inversion should be smaller than its Coxeter length. 

\end{itemize}

\section{Definition and some observations}
\begin{definition}[Weighted word length]
\label{def:wwl}
Let $S$ be a set of generators of group $G$. Let $\omega$ be a bounded function $\omega:S \rightarrow \mathbb{R}^+$. The weighted word length of a group element $g \in G$ is defined as 
\[l_{S,\omega}(g):=inf \{\sum_{i=1}^{t}{\omega(g_i)} \mid g_1^{\epsilon_1}g_2^{\epsilon_2} \hdots g_t^{\epsilon_t}=g, g_i \in S, \epsilon_i = \pm 1\}.\]
Weighted word length of the identity $e$ is $0$. 
\end{definition}

  



From the Definition~\ref{def:wwl}, the following observations are obvious.

\begin{lemma}[L1]
$l_{S,\omega}(g) = l_{S,\omega}(g^{-1}).$
\end{lemma}

\begin{lemma}[L2]
$l_{S,\omega}(g h) \leq l_{S,\omega}(g) + l_{S,\omega}(h).$
\end{lemma}

\begin{lemma}[L3]
$l_{S,\omega}(g h) \geq l_{S,\omega}(g) - l_{S,\omega}(h).$
\end{lemma}

\begin{lemma}[L4]
Let $\omega(t_{i,i+1})$ be $k$. Let $bp(g)$ be the number of breakpoints in $g$. Then,
\[ bp(g)/2 \leq l_{S,\omega}(g) \leq k \newl_C(g).\]
\end{lemma}

The upper bound on $l_{S,\omega}(g)$ is obvious from the definition of weighted length. The lower bound follows from the fact that an inversion can reduce the number of breakpoints of $g$ by at most 2. Thus at least $bp(g)/2$ edges are needed to get to $g$ from $e$. Since the cost of each edge is at least $1$, the minimum cost of a path to $g$ must be $bp(g)/2$.
\section{Approach}
\label{sec:app}

Let $P$ be a presentation for the group $S_n$ generated by $\Sigma$.
\[ P = \{\Sigma \mid R_1, R_2 \hdots R_k \}\]
Suppose $R_i$  is written as $e = R_i = uv$, which means $u = v^{-1}$. Suppose $l_{\Sigma,\omega}(u) < l_{\Sigma,\omega}(v)$. Let $w$ be an arbitrary word over $\Sigma$. If $w = x u y$ i.e., $u$ is a subword of $w$, then $w$ can be re-written as $w = x v^{-1} y$ and,
\[ l_{\Sigma,\omega}(x v^{-1} y) < l_{\Sigma,\omega}(x u y).\]

The first step in the approach taken in this work is to construct a library $\mathcal{L}_{<}$ of rewriting rules $u \rightarrow v$ where the rewriting rules are extracted from the relations in $P$ by splitting the relations $R_i$ into subwords of non-zero length. For example, coonsider the relation $s_1 s_2 s_1 = s_2 s_1 s_2$ where $s_1 = (1, 2)$ and $s_2 = (2, 3)$. Let the weight of $s_1$ and $s_2$ be $1$. Then $s_1 s_2 s_1 = s_2 s_1 s_2$ gives rise to the following rewriting rules :
\begin{align*}
s_1 s_2 s_1 s_2 &\rightarrow s_2 s_1, \\
s_1 s_2 s_1 s_2 s_1 &\rightarrow s_2. \\
\end{align*}

In a similar way, we construct a library $\mathcal{L}_{=}$, which consists of rewriting rules where both sides have the same weight. So, $s_1 s_2 s_1 \rightarrow s_2 s_1 s_2$ is a rewriting rule in $\mathcal{L}_{=}$.

To find a minimal weight representative for a group element $g$, we start with a word $w = g$. This is an easy step. For example, $w$ can be a word in Coxeter generators. Since each Coxeter generator is an inversion i.e., $C \subset \Sigma$, this makes sense - we are starting with a path in the Cayley graph of $S_n$ over $\Sigma$, and looking for shortcuts along this path to reach a lower weight path.

Scan $w$ to find a subword that has a smaller weight alternative in $L_{<}$. Keep rewriting $w$ until no subword to be replaced can be found.
At this point, we can navigate the $\mathcal{L}_{=}$ library to find equivalent words carrying the same weight and repeat the process of scanning for subwords.
The algorithm terminates when there exists no $u_i$ in $W$. 

This is most likely to result in a (local) minimum since scanning for a subword in $E$ might lead to several choices from which we always make a greedy selection. Disjoint substitutions seem to commute, however when faced with 2 potential substitutions with overlapping regions of influence, the decision affects future substitutions. Thus making arbitrary choices may lead one into a local ditch. 
% \section{Illustration}
% For $n = 4$, $\Sigma = \{(1, 2), (2, 3), (3, 4), (1, 3), (2, 4), (1, 4)(2, 3)\}$. 
% The weight vector is $\omega = \{1, 1, 1, 2, 2, 3\}$.

% \begin{table}[ht]
%   \centering
%   \begin{tabularx}{\linewidth}{lr}
%   1, 2, 1, 2, 1  & 2 \\
%   1, 2, 1, 2 & 2, 1 \\
%   2, 1, 2, 1 & 1, 2 \\
%   1, 2, 1, 3, 2, 1 & 6 \\
% \end{tabularx}
% \end{table}

\section{Test Results}
We have implemented the approach the outlined in section~\ref{sec:app} in GAP. The implementation is tested by comparing the results with that from igraph, which is a library of graph algorithms. For each permutation in $S_n$, we determine the minimal weighted path length using our implementation, and using igraph. The number of elements for which our algorithm failed to find the minimal path length is the number of failed cases (second column of table~\ref{tab: testresults}). The results are presented in table~\ref{tab: testresults}. 

\begin{table}[ht]
  \centering
  \begin{tabularx}{\linewidth}{lc}
  %\toprule
  \multicolumn{2}{l}{n = 5} \\
  \midrule
  Weights &  \\
  \midrule
  (1, 2, 3, 4) & 11 \\
  (1, 2, 3, 400) & 9 \\
  (1, 2, 300, 400) & 1 \\
  (1, 200, 300, 400) & 0\\ 
  \\
  \toprule
  \multicolumn{2}{l}{n = 6} \\
  \midrule
  (1, 2, 3, 4, 5) &  190  \\
  (1, 2, 3, 4, 500) & 188 \\
  (1, 2, 3, 400, 500) & 126 \\
  (1, 2, 300, 400, 500) & 18 \\
  (1, 200, 300, 400, 500) & 0 \\
  \end{tabularx}
  \label{tab: testresults}
  \caption{The first column is the vector of weights where the $i$th element if the weight assigned to $i+1$-inversion. The second column consists of the number of cases in which our implementation failed to find the minimal weight which was determined using igraph.}
\end{table}

\end{document}